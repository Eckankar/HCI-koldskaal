\section{Designvalg}
\subsection{Navigation}
En stor del af vores design er baseret på Ældresagens hjemmeside. Vi har
valgt at bruge deres menu-indeling med 4 hovedtabs: Medlemmer, For Frivillige,
Presse og Politik og Arbejdsliv. Nedenunder har vi en overordnet menu for hvert
emne, som hver åbner en mindre menu ude i venstre side. Dette valg gør, at vi i
princippet kan genskabe det meste af Ældresagens indhold med samme struktur som de
har, og det giver masser af plads til nyt indhold. 

Vi har på visse punkter valgt at lave ændringer i forhold til Ældresagens design, da
navigationen på deres side mildest talt er rodet, spredt og ikke-intuitivt. 

\begin{itemize}
\item Menuen bruger konsistente farver til at vise præcis hvor i menuen man befinder sig
\item Menuen er samlet til en menu i stedet for 3 visuelt forskellige menuer
\item Side-elementerne er altid samme steder, i modsætning til Ældresagen hvor de enkelte sider
      ofte er markant visuelt forskellige.
\end{itemize}

\subsection{Rabatter og tilbud}
Sektionen er ret ens, sammenlignet med Ældresagens. Vi har beholdt konceptet med et billede, en titel og en forklarende undertekst, der for os virker som om den giver det bedste overblik over kategorierne og deres indhold.

Højremenuen er fjernet, idet den komplicerede siden visuelt, og fjernede fokus fra det relevante indhold.
Noget af indholdet fra højremenuen burde nok beholdes (vigtigst er nok PDF'en over tilbudene), og den bedste måde at gøre det forestiller vi os ville være i en separat sektion under venstremenuen. 

\subsection{Kategorierne}
Kategorivisningerne følger samme struktur som den overordnede visning af kategorierne, hvilket burde hjælpe med at holde navigationen mere konsistent, og dermed med lidt held nemmere at bruge. 

\subsection{Panda}
Vi har ingen pandaer. Undskyld.
