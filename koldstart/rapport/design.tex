\section{Designvalg}
En stor del af vores design er baseret på Ældresagens hjemmeside, men vi har
bevidst ændret et par ting:

\begin{itemize}
\item Menuen bruger konsistente farver til at vise præcis hvor i menuen man befinder sig
\item Menuen er samlet til en menu i stedet for 3 forskellige menuer
\item 
\end{itemize}

\subsection{Rabatter og tilbud}
Sektionen er ret ens, sammenlignet med Ældresagens. Vi har beholdt konceptet med et billede, en titel og en forklarende undertekst, der for os virker som om den giver det bedste overblik over kategorierne og deres indhold.

Højremenuen er fjernet, idet den komplicerede siden visuelt, og fjernede fokus fra det relevante indhold.
Noget af indholdet fra højremenuen burde nok beholdes (vigtigst er nok PDF'en over tilbudene), og den bedste måde at gøre det forestiller vi os ville være i en separat sektion under venstremenuen. 

\subsection{Kategorierne}
Kategorivisningerne følger samme struktur som den overordnede visning af kategorierne, hvilket burde hjælpe med at holde navigationen mere konsistent, og dermed med lidt held nemmere at bruge. 

\subsection{Panda}
Vi har ingen pandaer. Undskyld.
