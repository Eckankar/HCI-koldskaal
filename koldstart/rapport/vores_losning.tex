\section{Fortolkning af opgaven}

I dette afsnit vil vi specificere hvordan vi har valgt at fortolke opgave
herunder de valg vi har været nødt til at tage for at kunne løse opgaven.
Derudover beskrives hvilke dele af hjemmesiden vi konkret har fået
implementeret.

Vi har valgt medlemstypen (guld, sølv, bronze) bliver afgjort ud fra hvor meget
du betaler i kontigent, da dette ikke var specifieret i opgaven og det blev
specificeret mundtligt at dette var en acceptabel fortolkning.

Vi har valgt i høj grad at kopiere menustrukturen fra ældresagen.dk. Vi har
dog gjort det tydeligere hvordan lagene af menuer afhænger af hinanden, idet
dette ikke var særligt intuitivt ud fra ældresagens hjemmeside. Vi har desuden
valgt at gøre meget lidt ud af de sider der ikke skal bruges til de konkrete
brugerscenarier samt at ofte lade links åbne en dialogboks hvori der står
``not implemented'' i stedet for at lave en side uden reelt indhold.

Med hensyn til den webshop vi skal implementere, har vi valgt at fortolke det
som at vi skulle tage en lignende funktionalitet til hvad der fandtes under
``Medlemstilbud'' $\rightarrow$ ``Rabatter og tilbud'', men med den yderligere
funktionalitet at man skulle kunne købe de tilbud man fandt herunder. Vi har dog
ikke implementeret funktionalitet til en indkøbskurv, men har i stedet fokuseret
på menuerne til at finde produkter, samt betalingsprocessen.

I det første brugsscenarie var der tale om at brugeren var ``Logget ind''. Vi
har i opgaven antaget, at denne login-funktionalitet er implementeret andetsteds
eller alternativt at der blot menes, at vedkommende har et medlemsnummer, som vi
nemt ville kunne tilføje et felt til et sted i betalingsformularen.

Vi har i opgaven valgt at betragte målgruppen som værende ældre og deres
pårørende.

Vi har brugt omtrent 15 timer per mand til implementation af siden samt
rapportskrivning.

