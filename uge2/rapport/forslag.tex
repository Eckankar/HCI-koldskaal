\section{Forslag til ændringer af jeres prototype}

Ud fra de ting vi har set på i opgaven, har der ikke været nogle grundlæggende
krav til forbedringer. Der har dog været en del mindre ting der sagtens kunne
kigges på - det var især følgende punkter som der skulle tænkes over:

\begin{itemize}
\item Understøttelse af mobile enheder
\item Understøttelse af zoom-funktionalitet
\item At der er tænkt over farveskemaet.
\item At der tænkes over muligheden for integration med andre systemer
\item At der tænkes over muligheden for en intern platform for organisationen
\end{itemize}

Af disse punkter er der allerede blevet tænkt over de fleste:
\begin{itemize}
\item Mobile enheder understøttes allerede til dels: Der er ikke noget der
direkte breaker, men man kunne godt forbedre det ved at lave et specifikt layout
til mindre skærme
\item Zoom-funktionalitet er så vidt vi har kunne teste understøttet. Det
breaker dog muligvis i andre browsere end Google Chrome. Dette skal naturligvis
undersøges og eventuelt forbedres
\item Farveskemaet kan sandsynligvis forbedres en del. I mange situationer vil
mange farver være forstyrende - men idet en stor del af vores målgruppe er
teknisk svage brugere, har vi valgt at gøre det meget tydeligt, hvori
menustrukturen man befinder sig
\item Der er ikke blevet tænkt over integration med anden software, idet det
lå udenfor opgavens rækkevidde. Det har desuden kun et indirekte HCI-aspekt og
vil derfor ikke blive prioriteret.
\item Det vil naturligvis heller ikke blive prioriteret at tænke over en intern
platform for organisationen, selvom det i et mere realistisk scenarie ville være
oplagt at gøre det.
\end{itemize}

Derudover fandt vi en række mere konkrete ting, som ikke ville kræve nogen
speciel omorganisering for at få implementeret:

\begin{itemize}
\item Der skal være en mere tydelig forklaring på forskellene, samt fordele og
ulemper ved de forskellige medlemstyper.
\item Printvenlig udgave af underkategorierne
\item Søgning (inkl. indenfor kategorierne)
\item Filtre på events (dato/forestilling/\ldots)
\end{itemize}
