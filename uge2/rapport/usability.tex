\section{Usabilityaspekter}
De vigtigste ting at tænke på vedrørende usability i vores webshop kommer fra to
betragtninger: de aspekter der er vigtige for webshops generelt, samt de hensyn vi
specielt skal tage, fordi vores målgruppe generelt er ældre og ikke har erfaring med
computere.

Bogen giver en liste af 12 designprincipper (Benyon s. 90).
De vigtigste for vores webshop er: 

\subsection{Synlighed}
Synlighed af vigtige funktioner er altid vigtigt i webshops, da selv de mindste forhindringer
kan betyde mistede salg. Her er det endnu vigtigere, da ældre ofte har dårligere syn.

\subsection{Konsistens}
Det er vigtigt at siden holder sig konsistent i sin virkemåde, da selv små ændringer fra side til side
kan give store problemer for ældre, der allerede har lært at bruge siden. Som det bliver diskuteret i
Further thoughts, Benyon s. 91, er der forskel på fysisk konsistens og konceptuel konsistens. Begge dele
er vigtige i vores webshop, da vi gerne vil have at brugerne skal lære så lidt som muligt for at bruge den.
En anden ting som bogen ikke diskuterer, men som er vigtigt for vores målgruppe, er at webshoppen helst skal
være konsistent over tid, så brugerne ikke hele tiden skal have opdateret deres mentale model af systemet.

\subsection{Genkendelse}
Hvor det er muligt, kan vi med fordel bruge metaforer, der giver det rigtige indtryk af hvordan siden virker.
De mest udbredte er funktionerne "Indkøbskurv" og "Gå til kassen".

\subsection{Navigation}
Dette er den vigtigste del af webshoppen (og af de fleste hjemmesider). Hvis navigationen ikke tydelig og
indeholder klare vejledninger, kan det betydeligt hindre brugen af webshoppen.

\subsection{Tilbagemeldinger}
Dette punkt er vigtigt for alle webshops. Brugeren har brug for at vide hvordan systemet reagerer på hans
handlinger, og hvilken tilstand systemet har, for at vide om han opnår det ønskede resultat. Der må aldrig
være tvivl om hvad der er i indkøbskurven, eller om at webshoppen er igang med at arbejde når en handling
udføres.

\subsection{Stil}
Da vores målgruppe primært er såkaldt "teknisk svage", gælder det om at holde designstilen så simpel
som muligt. Animationer, bevægende elementer osv. skal holdes på et minimum.

\subsection{Selskabelighed}
Fordi vores målgruppe er ældre mennesker, bør der være fokus på at systemet optræder venligt og respektfuldt.
Det er værdier som ældre mennesker generelt sætter mere pris på end resten af befolkningen, og det er vigtigt
for at hjemmesiden giver et godt indtryk.
En interessant ide fra Alan Cooper (Benyon s. 91, Box 4.3) er, at nye medier bliver opfattet som en
person, så man bør designe sit software til at opføre sig som en person folk kan lide. Det kunne komme til udtryk
gennem venlige, personlige beskeder (automatisk indsættelse af brugerens navn) eller, til et større system, en "guide"
til siden der har et menneskeligt ansigt. 


