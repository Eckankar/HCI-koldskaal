\section{PACT-analyse}

\subsection{Personer}
Den primære målgruppe for hjemmesiden er:
\begin{itemize}
\item pensionister: 60+ år, teknisk svage, svagtseende / dårlig hørelse, dårlig fingerfærdighed, langsomme til at lære.
\item pårørende: 20-60 år, teknisk meget svingende,
\item ansatte: 20-60 år, teknisk middel-svage,
\end{itemize}

\subsection{Aktiviteter}
Siden vil skulle bruges til følgende aktiviteter:
\begin{itemize}
\item Vedligeholdelse af hjemmesiden (ansatte)
\item Køb af produkter/tilbud (medlemmer)
\item Oprettelse af medlemsskab (pensionister/pårørende)
\item Få nyheder / læse om pensionistsagen (pensionister/pårørende)
\end{itemize}


\subsection{Kontekster}
\paragraph{Den fysisk kontekst} for brug af hjemmesiden vil typisk være en enkelt person
siddende foran en stationær computer. For pårørende (og muligvis ansatte) kunne
man dog også forestille sig, at de ville bruge mobile enheder til at tilgå
siden, hvorved de fysiske omgivelser er langt mere uforudsigelige.

\paragraph{De sociale kontekster} for siden er også meget enkeltstående -
man vil typisk være alene om at tilgå siden som desuden det desuden ikke er
planlagt skal have et socialt aspekt. Man kan dog forestille sig at pensionister
ville få hjælp til at tilgå siden (enten direkte eller over telefon) eller
at folk sammen prøver at finde ud af hvilke tilbud der er relevante (men
ikke direkte snakker om hjemmesiden). Desuden er der den sociale kontekst de
ansatte i pensionistsagen har i forbindelse med arbejdet på siden.


\paragraph{Fra et organisationelt} synspunkt vil hjemmesiden i sig selv ikke
umiddelbart have en specielt stor indflydelse, da de ansatte i organisationen
sandsynligvis ikke vil bruge hjemmesiden direkte, men vil bruge andre
værktøjer til deres daglige arbejde. Den vil dog potentielt have en stor
indirekte indflydelse, idet den vil kunne fungere som den primære kontaktflade
mellem organisationen og dens medlemmer og potentielle medlemmer. Man kunne
dog også forestille sig at hjemmesidens funktionalitet blev udvidet nok, til
at de ansatte ville kunne organisere den primære del af deres arbejde omkring
siden. Derudover vil det naturligvis skulle være nødvendigt for de ansatte at
vedligeholde indholdet på siden.

\subsection{Teknologi}
Hjemmesiden skal ideelt set understøttes af de standardfølgende webbrowsere, samt andre
store browsere på almindelige stationære systemer. Desuden skal der tilstræbes
understøttelse af mere ``moderne'' enheder, herunder især mobile enheder såsom
smartphones eller ``iPads''.

Til dette formål vil det være en fordel, hvis selve indholdet og form i så høj
grad som muligt blev holdt adskilt, idet det på denne måde er nemmere at senere
ændre siden, hvis det skulle vise sig at den ikke virkede på bestemte systemer.

Desuden kunne man forestille sig, at det ville være praktisk med API'er til
andet software - både software benyttet af de ansatte og af andre hjemmeside
(samt eventuelt til medlemmerne på et senere tidspunkt).

Derudover skal der laves integration med allerede eksisterende systemer.
