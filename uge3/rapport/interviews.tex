\section{Kontekstuel undersøgelse ved interview}

Vores interviewproces startede med at vi udarbejdede et script vi kunne følge. Vi valgte at følge en struktureret til semi-struktureret approach til at foretage de faktiske interview, hvor vi udeladte visse spørgsmål hvor det viste sig at være relevant.\footnote{Som set i Benyon 7.3}

Grundet tidsbegrænsning og manglende passende mennesker i vores omgangskreds var det desværre ikke muligt at ramme målgruppen præcist, så vi forsøgte at tilnærme det til bedste evne. Grundet dette var der måske nogle af interviewees'ne der ikke helt havde et lige så begrænset skill-set som en gennemsnitlig pensionist, hvilket kunne skjule visse problemområder.

Ydermere var det ikke muligt at mødes med alle folkene der skulle interviewest, så visse af interviews'ne blev fortaget over telefonen, hvorved vi mistede en del information vi kunne have fået ved at se hvordan de præcist reagerede og forsøgte at navigere på websiden.

Af personer har vi interviewet følgende:

\begin{description}
    \item[U1] {
    % Daniel
        En 63-årig efterlønsmodtager. Bruger computeren et par gange om ugen, primært til netbank, email og lidt informationssøgning. Bruger internettet et par timer om ugen. Det var ikke muligt at mødes med U1 for at foretaget et interview, så det blev foretaget over telefon.
    }
    \item[U2] {
    % Kristoffer
    }
    \item[U3] {
    En 53-årig kvindelig læge. Bruger dagligt nettet til fx at checke mail, se røntgenbilleder osv.
    }
    \item[U4] {
    % Mathias
    }
\end{description}

Til interviewet fulgte vi følgende manuskript:

\begin{enumerate}
    \item Grundlæggende information:
    \begin{enumerate}
        \item Alder.
        \item Beskæftigelse? Hvis pensioneret, lavede personen før.
        \item Hyppighed af computerbrug.
        \item Antal timers internetbrug per uge.
        \item Hvad er vigtigt i forbindelse med brug af internettet?
        \item Har du brugt ældresagens hjemmeside før?
        \begin{enumerate}

        \end{enumerate}
    \end{enumerate}
\begin{enumerate}   
