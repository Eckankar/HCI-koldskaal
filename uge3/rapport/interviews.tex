\section{Kontekstuel undersøgelse ved interview}

Vores interviewproces startede med at vi udarbejdede et script vi kunne følge. Vi valgte at følge en struktureret til semi-struktureret approach til at foretage de faktiske interview, hvor vi udeladte visse spørgsmål hvor det viste sig at være relevant.\footnote{Som set i Benyon 7.3}

Grundet tidsbegrænsning og manglende passende mennesker i vores omgangskreds var det desværre ikke muligt at ramme målgruppen præcist, så vi forsøgte at tilnærme det til bedste evne. Grundet dette var der måske nogle af interviewees'ne der ikke helt havde et lige så begrænset skill-set som en gennemsnitlig pensionist, hvilket kunne skjule visse problemområder.

Ydermere var det ikke muligt at mødes med alle folkene der skulle interviewest, så visse af interviews'ne blev fortaget over telefonen, hvorved vi mistede en del information vi kunne have fået ved at se hvordan de præcist reagerede og forsøgte at navigere på websiden.

Af personer har vi interviewet følgende:

\begin{description}
    \item[U1] {
    % Daniel
        En 63-årig efterlønsmodtager. Bruger computeren et par gange om ugen, primært til netbank, email og lidt informationssøgning. Bruger internettet et par timer om ugen. Det var ikke muligt at mødes med U1 for at foretaget et interview, så det blev foretaget over telefon.
    }
    \item[U2] {
    % Kristoffer
        En 70-årig pensionist, tidligere taxachauffør. Bruger primært computeren til et par google-søgninger, men printer også tilbud og spiller spil. Han bruger computeren ca. 8 timer om ugen.
    }
    \item[U3] {
    % Sebastian
    En 53-årig kvindelig læge. Bruger dagligt nettet til fx at checke mail, se røntgenbilleder osv.
    }
    \item[U4] {
    % Mathias
    En 70-årig pensionist. Meget aktiv mand, som bruger nettet meget, til blandt andet mails, blogs og nyheder.
    }
\end{description}

\subsection{Manuskript til interview}

Til interviewet fulgte vi følgende manuskript:

\begin{itemize}
    \item Grundlæggende information:
    \begin{itemize}
        \item Alder.
        \item Beskæftigelse? Hvis pensioneret, lavede personen før.
        \item Hyppighed af computerbrug.
        \item Antal timers internetbrug per uge.
        \item Hvad er vigtigt i forbindelse med brug af internettet?
        \item Har du brugt ældresagens hjemmeside før?
        \begin{itemize}
            \item Hvis ja: Kunne du forestille dig at bruge Ældre Sagens hjemmeside?
            \item Hvis nej: I hvilken forbindelse brugte du siden?
        \end{itemize}
        \item Hvad kunne være interessant for dig på sådan en side?
        \item Ældre Sagen tilbyder blandt andre ting rabatter på koncerter, hjælpemidler og lignende. Kunne du forestille dig at du ville bruge siden i den forbindelse?
        \item Hvilken information ville du ønske at se fremhævet i forbindelse med køb af følgende ting?
        \begin{itemize}
            \item Koncertbillet
            \item Høreapparat
            \item Ferierejse
            \item Cirkusbillet
            \item Social udflugt
        \end{itemize}
    \end{itemize}
    \item Test af Ældre Sagens hjemmeside:
    \begin{itemize}
        \item Hvad er din første indtryk af siden?
        \item Forestil dig at du gerne vil være medlem af Ældre Sagen. Prøv at finde information om at blive medlem.
        \begin{itemize}
            \item Vurdér oplevelsen på en skala fra 1--10, hvor 1 er meget nem og 10 er meget svær.
            \item Kommentarer?
        \end{itemize}
        \item Prøv at finde siden med medlemstilbud.
        \begin{itemize}
            \item Vurdér oplevelsen på en skala fra 1--10, hvor 1 er meget nem og 10 er meget svær.
            \item Kommentarer?
        \end{itemize}
        \item Du ønsker at købe en billet til musicalen ``Sværdet i Stenen''. Prøv at finde information om dette.
        \begin{itemize}
            \item Vurdér oplevelsen på en skala fra 1--10, hvor 1 er meget nem og 10 er meget svær.
            \item Kommentarer?
        \end{itemize}
        \item Hvordan var den generelle oplevelse af at bruge siden?
    \end{itemize}
    \item Test af Pensionistsagens prototypehjemmeside:
    \begin{itemize}
        \item Hvad er din første indtryk af siden?
        \item Forestil dig at du gerne vil være medlem af Ældre Sagen. Prøv at finde information om at blive medlem.
        \begin{itemize}
            \item Vurdér oplevelsen på en skala fra 1-10, hvor 1 er meget nem og 10 er meget svær.
            \item Kommentarer?
        \end{itemize}
        \item Prøv at finde siden med medlemstilbud.
        \begin{itemize}
            \item Vurdér oplevelsen på en skala fra 1-10, hvor 1 er meget nem og 10 er meget svær.
            \item Kommentarer?
        \end{itemize}
        \item Du ønsker at købe en billet til en. Prøv at finde information om dette.
        \begin{itemize}
            \item Vurdér oplevelsen på en skala fra 1-10, hvor 1 er meget nem og 10 er meget svær.
            \item Kommentarer?
        \end{itemize}
        \item Hvordan var den generelle oplevelse af at bruge siden?
    \end{itemize}
\end{itemize}
