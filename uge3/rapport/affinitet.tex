\section{Fortolkning af interview data}

Vi skabte vores affinitets-diagram ved at omforme vores interview-resultater til små, korte sætninger der dækkede et lille emne. Disse blev da fordelt ud på små sedler, som vi derefter arbejdede med at gruppere.
Generelt fandt vi ret god sammenhæng i mange af dem, men der var visse der ikke helt var relevante.

Vedlagt i bilag ses resultatet af vores affinitets-diagrammering, inklusiv overskrifter på vores grupperinger. Desuden kan man se billeder fra vores diagrammerings-proces.

Vi lagde mærke til følgende trends i vores resultater:

\begin{itemize}
    \item Vores ændring til menuen fik generelt positivt feedback, men dog var farverne ikke så populære.
    \item Menuerne er forvirrende, men dette vidste vi i forvejen.
    \item Information om accessibility og om pårørende kan tage med i forbindelse med arrangementer er vigtigt og bør highlightes.
    \item Teksten bør være større, med en klar overskrift så den ældre nemmere kan forstå hvor de er.
    \item Eksempelkategorierne er ikke altid helt logiske (fx at koncerter ikke er under underholdning).
    \item Søgefunktionaliteten er meget vigtig og bør forbedres.
    \item Arrangementer bør laves i samme stil som webshoppen for at skabe uniformitet; mange søgte derover for at finde information om koncerter.
    \item En let og gennemskuelig URL der er tilgivende kunne være godt.
\end{itemize}
