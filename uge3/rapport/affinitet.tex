\section{Fortolkning af interview data}

Vi skabte vores affinitets-diagram (se bilag \ref{b:aff}) ved at omforme vores
interview-resultater til små, korte sætninger der dækkede et lille emne.
Disse blev da fordelt ud på små sedler, som vi derefter arbejdede med at
gruppere. Generelt fandt vi ret god sammenhæng i mange af dem, men der var
visse der ikke helt var relevante.

Vedlagt i bilag ses resultatet af vores affinitets-diagrammering, inklusiv
overskrifter på vores grupperinger. Desuden kan man se billeder fra vores
diagrammerings-proces. vi identificerede følgende grupper af udsagn:

\begin{enumerate}
    \item Ting vedr. Ældre Sagens hjemmeside
    \item Hvilke motivationer folk har for at besøge hjemmesiderne
    \item Udsang omhandlende informationssøgning
    \item Tekniske fejl og mangler
    \item Urelaterede udsagn (\texttt{/dev/random})
    \item Brugergrænsefladen
    \item Navigation af hjemmesiden
    \item Ting vedr. Pensionistsagens prototype
    \item Hvilken type information folk i målgruppen søger
\end{enumerate}

Gennem analysen fandt vi frem til følgende punkter der enten var
problematiske eller vigtige for de personer der var blevet interviewet:

\begin{itemize}
    \item Vores ændring til menuen fik generelt positivt feedback, men dog var farverne ikke så populære.
    \item Menuerne er forvirrende, men dette vidste vi i forvejen.
    \item Information om accessibility og om pårørende kan tage med i forbindelse med arrangementer er vigtigt og bør fremhæves.
    \item Teksten bør være større, med en klar overskrift så den ældre nemmere kan forstå hvor de er.
    \item Eksempelkategorierne er ikke altid helt logiske (fx at koncerter ikke er under underholdning).
    \item Søgefunktionaliteten er meget vigtig og bør forbedres. Bl.a. skal søgning være let at tilgå og man skal kunne finde alle de nødvendige informationer. Det var fx ikke muligt at finde ``Stenen i sværdet'' på Ældre Sagens hjemmeside gennem søgning.
    \item Arrangementer bør laves i samme stil som webshoppen, eller muligvis slås sammen, for at skabe uniformitet; mange søgte derover for at finde information om koncerter.
    \item En let og gennemskuelig URL der er tilgivende kunne være godt. Fx bør hjemmesiden være tilgængelig via \texttt{www}-subdomænet idet dette er et standardidiom, der for mange mennesker kendetegner en hjemmeside.
\end{itemize}

Bilag \ref{b:aff_billeder} viser billeder fra affinitetsdiagrammeringen. Det
viste sig at være svært at finde enten saks eller skæremaskine på DIKU,
så vi måtte drage over på HCØ for at finde skære-/klipperedskaber.
