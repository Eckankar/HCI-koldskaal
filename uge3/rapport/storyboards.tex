\section{Detaljering af vision med storyboards}

\subsection{Storyboard 1}
Her følger vi igen Gerda, der skal til et arrangement. Hun trykker på kalenderen på forsiden. 
Hun kommer ind under kalendervisningen, og vælger det hun gerne vil til. Her kan hun bestille billet til det valge arrangement.


\subsection{Storyboard 2}
Gertrud vil gerne til et computerkursus. Hun klikker på arrangementer, så der kommer en kategorivisning af de forskellige arrangementer.
Hun vælger kategorien "Kurser", og finder et kursus i pc-kørekort. Her står der information om hvornår det er, og hun kan bestille en plads på
kurset.

\subsection{Arbejdets forløb med visioner og storyboards}
Vi følte ikke, at vi havde nogle af de problemer som bliver nævnt af Holtzblatt (s. 220 i Rapid Contextual Design):
vores gruppe er lille og vi var alle 3 involverede i processen. Til gengæld følte vi, at vores opgave var så veldefineret,
at visionerne ikke var til megen nytte. Vores primære del af siden er webshoppen, som man i sig selv fastlægger meget på siden,
og derudover havde vi allerede fastlagt navigationen på siden i koldstartsopgaven. Derfor endte vores visioner med at minde meget
om vores storyboards. 

