\section{Alternative visioner}
Vi har fulgt visioneringsprocessen, som beskrevet i \cite[Kapitel 11]{Holtzblatt2005}, så tæt som muligt. Vi har ikke fulgt rollerne som beskrevet \cite[s. 214-216]{Holtzblatt2005}, da der så kun ville være 1 person til rent faktisk at fortælle historien. 


\subsection{Vision 1}
I vores første vision kom vi primært til at fokusere på en arrangementskalender samt dens betydning for resten af siden.
Vi udforskede kalenderens sammenhæng med både vores frontend (det som pensionisterne ser) og vores backend (det som pensionistsagens medarbejdere ser).
Derudover har vi forestillet os nogle nye features, f. eks. login med NemID, en personlig kalender til hver bruger og flere forskellige betalingsformer.

\subsubsection{Fordele}
\begin{itemize}
\item NemID kan gøre betaling nemt 
\item Kalenderen giver bedre overblik over arrangementer
\item Personlig kalender gør det muligt at huske hvilke arrangementer man skal til
\item Nye tilbud på forsiden kan give motivation til at vende tilbage oftere
\end{itemize}

\subsubsection{Ulemper}
\begin{itemize}
\item NemID gør det besværligt at logge ind
\item Visse betalingsformer kan være krævende at understøtte
\item Forsiden kan nemt blive for rodet
\item Personlig kalender kan nemt komme i vejen, hvis brugeren ikke bruger siden ofte
\end{itemize}

\subsection{Vision 2}
Her endte vi med at tage fat i selve webshop-delen af siden. Der er masser af features som de fleste moderne webshops har, f. eks. foreslåede varer ud fra tidligere søgninger og køb, produktanmeldelser, gavekort, produktsøgning osv. Desuden er der reklamer på forsiden, der promoverer interessante varer, eventuelt med tidsbegrænsede tilbud.

\subsubsection{Fordele}
\begin{itemize}
\item Andre brugeres anbefalinger i produktanmeldelser kunne gøre brugeren mere tryg ved køb
\item Gavekort kunne hjælpe på salget
\item Søgning gør det lettere at finde det man leder efter
\item Indlejrede videoer af produkterne i aktion kan motivere yderligere køb
\item Relaterede søgninger fra andre brugere kan inspirere brugere til at kigge på andre varer
\end{itemize}

\subsubsection{Ulemper}
\begin{itemize}
\item Brugere har en tendens til ikke at kigge på elementer, der ligner reklamer \cite{Grue}
\item For mange features kan gøre købsprocessen uoverskuelig
\item Relaterede søgninger virker ikke altid godt ved små udvalg af varer
\end{itemize}

\subsection{Konsolideret vision}
Ud fra evalueringen af de to foregående visioner har vi lavet en konsolideret vision. De to visioner overlapper ikke væsentligt, så det var muligt at sætte dem sammen uden at fjerne så meget. Vi droppede NemID, og alt efter hvor stor webshoppen er bliver relaterede søgninger formegentligt også droppet. Vi har også valgt store eventreklamer fra på forsiden for at holde den overskueligt, og i stedet valgt en liste over nye tilbud, som ligner indhold frem for reklamer.
