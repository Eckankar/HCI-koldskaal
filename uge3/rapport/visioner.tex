\section{Alternative visioner}
Vi har prøvet at følge de guidelines, som er nævnt i kapitel 11 i Rapid Contextual Design, hvor det har været muligt.
Da vi kun er 3 i gruppen, følte vi ikke at vi kunne påtage os så specifikke roller som de lagde op til, og vi deltog derfor
alle sammen ligeligt i visioneringen. Vi lavede 2 visioner: Gerda, der gerne vil vide hvilke arrangementer der går det næste stykke tid
og eventuelt melde sig til dem, og Gertrud, der gerne vil finde et kursus, hvor man lærer at bruge computere. Til sidst konsoliderede vi dem
til en vision, der mere viser hvordan de to eksempler hænger sammen. Visionerne kan ses i bilagene.

\subsection{Indflydelse fra affinitetsdiagrammeringen}
Det lykkedes os at indrage nogle af konklusionerne fra vores interviews i vores visioner. Vi valgte at lave meget af navigationen som billeder, da 
en af vores interviewpersoner foretrak at navigere via disse. Derudover har vi fokuseret meget på at integrere sektionen "arrangementer" og webshoppen,
fordi det var mere intuitivt for flere af vores interviewpersoner at finde koncerter osv. under denne sektion. 


\subsection{Vision 1}
I den første vision følger vi Gerda, en pensionist på 72 år. Hun vil gerne se hvilke arrangementer der går i pensionistsagen det næste stykke tid, og hvis hun finder noget hun godt kan lide, så vil hun gerne melde sig til eller købe en billet til det. På vores forside er der både en undermenu der hedder "arrangementer", samt en kalender der viser de kommende arrangementer. Menuen kan bruges til at få en kategorisk oversigt over hvad der går, mens kalenderen sender en direkte til det arrangement man klikker på. Når man er nået dertil er købsprocessen simpel.

\subsection{Vision 2}
Her ser vi Gertrud på 69. Hun vil gerne blive bedre til at bruge computere, så hun vil ind og se om der er et kursus der kan lære hende det. Der er to veje der leder til det samme mål: hvis hun går ind i "råd og guides" kan hun finde kategorien computerhjælp, og her kan hun se at der tilbydes et kursus. Hvis hun vælger "arrangementer" først, kan det samme kursus findes her. Hvis kurset går inden for det næste stykke tid, kan det desuden findes på kalenderen på forsiden. Denne vision lægger vægt på redundans, da flere af vores interviewpersoner havde nogle fornuftige idéer til at finde det de gerne ville, og derfor vil vi gerne understøtte så mange af dem som muligt.

\subsection{Konsolideret vision}
Vores konsoliderede vision prøver at vise, hvordan siden hænger sammen. Der er mange måder at opnå samme resultat, og selvom vi kun viser de delmængder af siden som vi har udforsket i de 2 andre visioner, så er det et gennemgående træk vi ønsker for siden.
