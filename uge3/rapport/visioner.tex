\section{Alternative visioner}
Vi har prøvet at følge de guidelines, som er nævnt i kapitel 11 i Rapid Contextual Design, hvor det har været muligt.
Da vi kun er 3 i gruppen, følte vi ikke at vi kunne påtage os så specifikke roller som de lagde op til, og vi deltog derfor
alle sammen ligeligt i visioneringen. Vi lavede 2 visioner: Gerda, der gerne vil vide hvilke arrangementer der går det næste stykke tid
og eventuelt melde sig til dem, og Gertrud, der gerne vil finde et kursus, hvor man lærer at bruge computere. Til sidst konsoliderede vi dem
til en vision, der mere viser hvordan de to eksempler hænger sammen.

\subsection{Indflydelse fra affinitetsdiagrammeringen}
Det lykkedes os at indrage nogle af konklusionerne fra vores interviews i vores visioner. Vi valgte at lave meget af navigationen som billeder, da 
en af vores interviewpersoner foretrak at navigere via disse. Derudover har vi fokuseret meget på at integrere sektionen "arrangementer" og webshoppen,
fordi det var mere intuitivt for flere af vores interviewpersoner at finde koncerter osv. under denne sektion. 


