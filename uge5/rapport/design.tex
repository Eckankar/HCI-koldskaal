\chapter{Visuelt design}

\section{Farver}
Vores webshop bruger farver, men kun i sammenhæng med navigationen af siden. I vores menu bruger vi farver til at vise hvilke hovedmenu eller "tabs"
der hænger sammen med den nuværende side, i stil med gestaltloven om lighed \cite[s. 336]{Benyon2010}. Til selve indholdet bruger vi sort tekst
på hvid baggrund, som er bedst til at viderebringe information \cite[s. 430]{marcus}. Det eneste sted vi bruger farver til at henlede opmærksomhed er
til links, som er blå og understregede ifølge normale webkonventioner, samt på tilmeldingssiden, hvor røde stjerner indikerer at visse felter skal udfyldes
for at fortsætte, hvilket også er forklaret i bunden af siden. Dette følger anbefalingen i designregel 5 \cite[s. 344]{Benyon2010}.

\section{Ikoner}
Navigationen til bestemte produkter og produktkategorier foregår primært med billeder der virker som links. Dette har vi valgt, da vores brugerundersøgelse
pegede i retning af, at vores målgruppe at navigere ved at bruge billeder \cite{os}. Billederne illustrerer enten selve produktet eller en generel situation
der skal lede tankerne hen på bestemte situationer, der er relevante for enten et produkt eller en kategori. 
Vores billeder følger mange af punkterne fra Hortons' ikon-tjekliste \cite[s. 327]{Benyon2010}, mens andre af punkterne beskæftiger sig med ikoner generelt
og derfor er irrelevante. De vigtigste punkter som vi prøver at overholde er forståelig, familiaritet og mindeværdighed. 

\section{Hukommelse}
login
chunking
drop down / picklists

\section{Wizards}


\section{Gestaltlove}


