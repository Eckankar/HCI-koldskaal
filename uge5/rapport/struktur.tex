\chapter{Overordnet struktur og design}

En af de ting vi valgte at ændre i prototypen var den måde vores URLs
bliver genereret på, så de bliver mere overskuelige for brugeren \cite[s.
385]{Benyon2010}. Fx er det ikke særligt hensigtsmæssigt at URLen til ``Bliv
medlem'' er \url{http://pensionist.coq.dk/?level1=0&level2=1&level3=0}. Dette
er nu blevet ændret til \url{http://pensionist.coq.dk/medlemsskab}.

Siden er også blevet ændret således at forsiden for ``Rabatter og tilbud''
indeholder et generelt overblik hvorfra man kan komme videre til de enkelte
underpunkter \cite[s. 386]{Benyon2010}. Siden imødekommer nu også folk der
er mere søgefokuserede idet man fra ethvert sted på siden nemt kan tilgå
søgefunktionaliteten \cite[s. 386]{Benyon2010}.

Ved organiseringen af indkøbsportalen har vi valgt at bruge flere forskellige
klassificeringsskemaer som beskrevet i \cite[s. 392--394]{Benyon2010}.
Gennem kalenderen har vi en kronologisk organisering af arrangementerne.
En mere ``task''-baseret organisering er dog også tilgængelig gennem
kategorivisning, og her kan man også finde alle de ting der ikke er
tidsbestemte (bøger, film etc.).

Idet man kan tilgå en delmængde af tilbuddene gennem flere forskellige
navigationsstier (enten gennem kalenderen eller kategorierne) er denne del af
hjemmesiden altså organiseret mere som et netværk end en strengt hierarkisk
struktur \cite[s. 396]{Benyon2010}. Når man har nået et enkelt produkt
skifter det dog til en sekventiel organisering så selve købsprocessen bliver
forenklet i form af en slags ``wizard''.

Sidens globale navigation er implementeret i form af faneblade. Når
man tilgår undersiderne for man efterfølgende adgang til en sekundær
samling af faner. Dette burde gøre det nemmere at hoppe højere op i
navigationstræet eller skifte til en anden af hjemmesidens hovedkategorier
\cite[s. 402]{Benyon2010}. Under førnævnte købsprocess understøttes
brugeren af en slags statusbar der viser hvor langt man er nået så stien
bliver mere tydelig \cite[s. 403]{Benyon2010}.

Slutteligt er har søgning som sagt fået en større prioritet på
hjemmesiden fordi det for mange mennesker er blevet en tilvant måde at finde
informationer på hjemmesider \cite[s. 405]{Benyon2010}. Vi har imidlertid
valgt at tilsidesætte det tekniske aspekt med at implementere søgningen, så
i prototypen vil den udelukkende fremstå som et visuelt element i designet.
