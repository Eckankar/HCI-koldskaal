\section{Papirprototype}

Ud fra visionerne i \cite{os} identificerede vi følgende yderligere
elementer til interfacet:

\begin{itemize}
    \item Personlig kaldender indeholdende kommende arrangementer for brugere
          der er logget ind
    \item Kortfattet kalender indeholdende kommende arrangementer
    \item Stor kalender med arrangementer
    \item Søgeboks
    \item Gavekort
    \item Forsidetilbud
    \item Videodemonstration af produkter
    \item Anmeldelser
\end{itemize}

Analysen gav ikke anledning til at fjerne nogle ting fra den eksiterende
prototype. Ud fra disse ting lavede vi en ``first round''-papirprototype
som beskrevet i \cite{Holtzblatt2005}. Papirprototypen kan findes i bilag
\ref{b:papirprototype}.

Det første brugsscenarie fra koldstartsopgaven omhandler en eksisterende
bruger der er logget ind på Pensionistsagens hjemmeside. Hun ønsker at tage
til en koncert med Brahms. Her hjælper kalenderen med at finde relevante
tilbud. Gennem kalenderen kan hun finde den måned hvor hun ønsker at finde
en koncert og gå direkte til købssiden. Herfra kan hun finde praktiske
informationer, oplysninger om hvad andre brugere på siden synes og hun kan
bestille billeterne.

I det andet brugscenarie er brugeren ikke endnu ikke et medlem, men ønsker at
tegne et medlemsskab. I hjemmesidens sidebar er der nu for brugere der ikke
er logget ind en boks der linker direkte til tilmeldingssiden. Brugeren får
på tilmeldingssiden en kort oversigt over de forskellige medlemstyper som
Pensionistsagen tilbyder. Endvidere er der et link så brugeren kan få mere
uddybende information omkring dette.

Overordnet set har vi valgt at bevare grundelementerne i vores design.
Kun enkelte, nye elementer er blevet tilføjet for enten at forbedre
brugeroplevelsen eller eller give ekstra features jf. visionerne fra
\cite{os}.
