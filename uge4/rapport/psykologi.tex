\section{Redegørelse for psykologiske aspekter}

Psykologiske aspekter er vigtige at have med i sine overvejelser, når man
designer interaktive systemer. Brugere er normalt gode til at forstå og
interagere med systemer, men de har også visse basale begrænsninger, som
man bliver nødt til at tage højde for når man designer brugergrænseflader
\cite[s. 531]{Benyon2010}. Her ser vi på nogle af de vigtigste aspekter i
forhold til vores indkøbsportal.

\subsection{Hukommelse}
Hukommelse kan groft set deles op i to dele -- arbejdshukommeles og
langtidshukommelse \cite[s. 536]{Benyon2010}. Arbejdshukommelsen indeholder
den information, so man bruger til at udføre ens nuværende opgave.
Menneskers korttidshukommelse er begrænset \cite[s. 536]{Benyon2010}, så
hvis man bliver forstyrret midt i en process, kan den relevante information
hurtigt blive erstattet med noget andet. For at forebygge dette bør sekvenser
af handlinger, f.eks. køb af en vare, være ledsaget af klare instruktioner
og visuelle påmindelser om den igangværende handling. Derudover bør brugere
skulle huske så lidt information som muligt udenad for at bruge siden. Et
godt eksempel herpå er automatisk login, automatisk udfyldning af adresse
etc.

\subsection{Opmærksomhed}
Det åbenlyse problem med opmærksomhed er at fastholde den. I webdesign er
ventetid en vigtig ting at få minimeret \cite[box 22.7]{Benyon2010}, da det
let kan få folk til at miste koncentrationen.

Det alment kendte fænomen, bannerblindhed, er også vigtigt at undgå, jf.
\cite{Grue}. Hvis sidens indhold ligner reklamer, bliver det ikke set af
brugere der er vant til at bruge internettet. Selvom vores målgruppe er
teknisk svage, er det stadig en god ting at have i mente.

En af de vigtigste aspekter af opmærksom er såkaldte "action slips", som er bestemte typer fejl
der hyppigt bliver begået \cite[s. 553]{Benyon2010}. Den hyppigste fejl blandt disse er "Storage failures",
hvor en bruger glemmer at han/hun har udført en handling, som han/hun derfor udfører igen. En anden type fejl
der er relevant for en webshop er "Subroutine failures", hvor folk glemmer et trin i en sekvens af handlinger
de er igang med at udføre. Disse kan minimeres ved hele tiden at give folk klare instruktioner for hvad de
skal gøre og hvordan systemets tilstand er, hvilket f. eks. kunne gøres med en "wizard" \cite[s. 553]{Benyon2010}.

\subsection{Affekt}
De fleste områder der nævnes inden for affekt er interessante, men er enten upassende, eksperimentielle eller langt uden for vores rækkevidde at
implementere. Det er klart, at tiltag der kræver mere teknologi end en webbrowser ikke fungerer i en webshop. En idé der kunne være interessant
at udforske er at have en "virtuel person" på hjemmesiden, som guider brugeren videre og som kan svare på simple spørgsmål og udvise simple følelser.
Som nævnt af Benyon er det tvivlsomt om det forbedrer brugerens oplevelse af systemet, da det både kan virke irriterende og da folk har en tendens
til at forvente mere af systemet end det kan levere. \cite[s. 571]{Benyon2010}.

\subsection{Kognition}
Den 

