\section{Vision og prototype}

De overordnede ideer i vores design omhandler primært webshopdelen af
hjemmesiden, da den udgør størstedelen af funktionaliteten. Ideerne blev
udledt af visionerne fra \cite{opgave3}. Relaterede søgninger blev i den
konsoliderede vision nævnt som en ting der kunne være god at have, men
kunne være problematisk. Relaterede søgninger er dog også blevet droppet
pga. tidsmangel og fordi den er en af de mindre vigtige ting. Det blev heller
aldrig fuldt ud specificeret hvordan de relaterede søgninger skulle fungere
og hvilke parametre der skulle være afgørende.

Søgning i sig selv bliver stadigvæk opfattet som en vigtig ting, men på
grund af kompleksiteten af en søgefunktion fremgår den i prototypen kun som
et søgefelt uden tilhørende funktionalitet. Ideelt havde søgning været
implementeret, men der var desværre heller ikke tid til dette.

Indlejrede videoer er en del af produktbeskrivelsen, så implementation af
dette kan overlades til administrationsdelen af hjemmesiden.

Visionen indeholdt elementer til administration af hjemmesiden, men dette er
ikke blevet implementeret idet Pensionistsagens ansatte udgør en så lille
del af brugergruppen at det ikke kan betale sig at få i orden til prototypen.
Implementation af dette ville i øvrigt udgøre en væsentlig omskrivning af
koden idet det meste indhold er statisk genereret.

Kalenderfunktionaliteten var en vigtig del af vores vision idet kalenderen
gør det væsentligt nemmere for brugerne at finde arrangementer i netop den
tidsperiode, som brugeren ønsker.

TODO: Skriv noget om afvigelser fra papirsprototypen i \cite{opgave4} her!
