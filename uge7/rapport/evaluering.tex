\section{Heuristisk evaluering af prototypen}
Da vi ikke har kunnet finde nok brugere til at lave en rigtig brugertest i tide, har vi i stedet udført en
heuristisk evaluering af vores endelige prototype. Metoden, som beskrevet af Benyon, er at flere usabilityeksperter
hver især gennemgår prototypen, mens de finder usabilityproblemer i forhold til bestemte forudvalgte heuristikker
\cite[s. 228]{Benyon2010}. Som heuristik brugte vi designprincipperne som nævt af Nielsen i hans beskrivelse af "Discount usability engineering" \cite[tabel 2.2]{nielsen}:

\begin{itemize}
\item Synlighed af systemets status
\item Sammenhæng mellem systemet og den virkelige verden
\item Bruger kontrol og frihed
\item Konsistens og standarder
\item Forhindring af fejl
\item Brugere skal genkende i stedet for at huske
\item Fleksibilitet og nemt at bruge
\item Æstetisk og minimalistisk design
\item Hjælp brugere komme videre efter fejl
\item Hjælp og dokumentation
\end{itemize}

Som nævnt af Benyon bør man aldrig evaluere sit eget design, medmindre der ikke er nogle alternativer
Det har vi imidlertidigt ikke haft, og derfor har alle gruppemedlemmer deltaget i evalueringen. Det kan  
medføre både falske positiver og en overfokusering på små obskure fejl \cite[s. 229]{Benyon2010}. 

Selve evalueringen foregik individuelt, og vi tog selv noter af forløbet. Det gav følgende liste af usabilityproblemer:

\begin{itemize}
\item Synlighed forsøges vist fx ved box i venstre side der viser om man er logget in (kunne godt være klarere).
\item På butiksforsiden er kategoriikonerne mindre end andre steder, hvilket giver anledning til lidt inkonsistens.
\item Links er ikke altid klart markerede.
\item Brugen af billeder bør give mindre tankearb. i forb. med navigationen.
\item Brugeren kan komme til at afbryde et køb - ingen advarsel.
\item Menu er for kompliceret.
\item Shopforside måske begyndt at blive for fyldt.
\item Betalingssiden kunne nok godt bruger lidt mere hjælpetekst.
\item Links under ``Medlemstilbud'' har ikke samme farve som links på resten af hjemmesiden.
\item Ved tilmelding kunne det være hensigtsmæssigt at vise personen hvilke oplysninger de havde givet for at sikre mod tastefejl. Hvis formularen bliver udfyldt med forkert data ville brugeren nok undre sig over hvorfor han/hun aldrig har hørt noget. Samme problem med køb. Der er ingen "overview" side med en endelig "confirm" knap.
\item Farverne i menuen passer ikke godt med indholdet.
\item Medlemsservice er ikke dækkende for ikke-medlemmer, forslag: lav en genvej til bliv medlem i menubaren.
\item Formen i "bliv medlem" skal tjekke om oplysningerne er gyldige.
\item Punkterne i ens personlige kalender bør være links.
\item I første side af købsprocessen er knapperne med antal og køb for små.
\end{itemize}

Da der ikke er blevet fundet særligt mange problemer, som til dels også skyldes at meget af funktionaliteten ikke er implementeret, er det ikke nødvendigt at rangere
dem i forhold til alvorlighed, da alle problemerne burde kunne løses tilfredsstillende. 
