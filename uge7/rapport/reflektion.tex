\section{Reflektion over manglende brugertests}

I realiteten ville det at vi havde foretaget en brugertest nok ikke have
påvirket vores prototype, i og med at kurset er slut.

Dog, ville vi, såfremt vi havde været ude i den virkelige verden have
mistet feedback på, om hvorvidt de ændringer vi havde lavet havde været
konstruktive i forbindelse med udvikling til vores målgruppe. Vi kunne fx
spekulere at vores komplicering af shopforsiden muligvis kan føre til for
meget unødig forvirring, i forhold til hvor meget det gavner.

Det er næsten garanteret guld man går glip af, når man udelader at høre
ens målgruppes oplevelser ved brug af grænsefladen.

Årsagen til at vi ikke fik udført brugertests skyldes nok primært manglende
kommunikation i gruppen. Gruppens medlemmer havde ikke i ordentlig tid
opfattet at ugeopgave 6 og ugeopgave 7 havde samme afleveringstidspunkt, eller
at arbejdet med ugeopgave 7 krævede involvering af tredjeparter. Således
kom vi altså for sent i gang med at finde mulige kandidater til brugertests
inden for den tid vi havde til rådighed hvis vi også skulle have nået at
bearbejde de data vi ville have fået ud af sådanne tests. Endvidere havde vi
det problem at prototypen blev sent færdigt færdigtudviklet hvilket også
gjorde det sværere at arrangere tests med brugere.
